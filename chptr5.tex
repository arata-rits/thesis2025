%/**********************************************************************/
%		第2章:LED照明によるフリッカノイズ現象とその関連研究
%/**********************************************************************/
\chapter{結論}
\label{lbl_chptr2}





%/**********************************************************************/
%		イントロの章
%/**********************************************************************/

.............................................................................

\section{まとめ}
本論文ではデジタルサイネージに対する情報付加手法の提案と,制作した制御プログラムによるサイネージの点灯を行ってきた.ダイナミック点灯によるステゴパネルの実装に関しては,大幅にハードウェア量を削減できた.しかしながら,ダイナミック点灯等の特性上,本論文で取り上げた手法では,従来のステゴパネルと同等の精度を実現する事は困難であった.その一方で.肉眼視認時に,カメラ視認時と異なる画像を表示するステゴパネルなど,ステゴパネルの新たな可能性を見い出すことができた.色差の少ない$8\times8$程の低解像度ドットマトリクスLEDを用いることで,濃淡の少ない仕様を実現できると考えられる.
トリケラパネルに関して,カラーセロハンによる簡易的な色の抽出ではQRコードの読み取りは困難であった.その一方で,簡単な図柄に対してはカラーセロハンによる色の抽出で肉眼で視認できた.今後は,画像処理等の抽出技術による補完で実用的なものになると考えられる.


\section{今後の展望}

\subsection{ステゴパネルの展望}

ステゴパネルの展望として,小型化,軽量化,低消費電力化が挙げられる.従来のステゴパネルでは,基盤作成時に片面実装により制作されていた.よって両面実装を取り入れることで,ある程度の小型化が可能になると考える.また制御信号を振り分けるためのシフトレジスタの数が多く,仕様の実現上,減らすことが困難であるため,実用化に向けた小型化を考える際には,基となる回路を見直す必要がある.


\subsection{トリケラパネルの展望}
トリケラパネルの展望として,高解像度化によるイラストの表示が挙げられる.セロハンによるフィルタでも肉眼でイラストが確認できることから脱出ゲームなどのエンターテイメント領域での活用が期待できる.また,カメラアプリによる画像処理的なアプローチでQRコードの読み取りが可能になると考える.例えば,色の抽出を行った画像を画像処理により2値化し,モノクロ画像としてQRコードを読み込む方法である.従って,今後は色の抽出とQRの読み取りを行うカメラアプリの開発に注力したい.







%/**********************************************************************/
%		関連研究の章
%/**********************************************************************/