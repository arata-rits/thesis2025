%/**********************************************************************/
%		第4章:
%/**********************************************************************/
\chapter{検出性能の比較実験}
\label{lbl_chptr2}

本章では,従来のモノクロQRコードとカラーQRコードの検出率による比較実験について述べる.はじめに,4.1節で実験手法について述べ,4.2節で実験結果について述べる.また,4.3節では実験に対する考察を述べる.


%/**********************************************************************/
%		イントロの章
%/**********************************************************************/



\section{実験手法}
実験に際して.10種類のカラーQRコード画像を用意した.これは.カラーQRコードの種類による検出への影響を調べる為である.用意した10種類のカラーQRコードが図\ref{fig:CQR_sample0_9}である.これらのカラーQRコードは,ランダムに生成した10文字の半角英数字を3種類用意し,先頭にサンプル番号を追加した計11文字の半角英数字をエンコードすることで作成している(表\ref{tab:sample_string})ここで,サンプル番号とはsample2であれば,数字の2を表す.また,画像の撮影には,apple社製のスマートフォン端末であるiPhone XRを使用した.搭載されている背面カメラの画素数は1200万画素である\footnote{iPhone XR - 技術仕様|\url{https://support.apple.com/ja-jp/111868}より引用}.

撮影距離について,20 cm, 40 cm, 60 cm, 80 cmの4種類から撮影を行った.撮影時の実験環境を図\ref{fig:experiment_situation}に示す.



\begin{figure}[H]
	\centering
	\includegraphics[width=1.0\linewidth]{pics/CQR_sample0_9.png}
	\caption{実験に使用したカラーQRコード画像}
	\label{fig:CQR_sample0_9}
\end{figure}





\begin{table}[]
	\centering
	\setlength{\tabcolsep}{10pt}
	\renewcommand{\arraystretch}{1.4}
	\caption{カラーQRコードサンプルに使用した文字列}
	\label{tab:sample_string}
	\begin{tabular}{|l|l|l|l|}
		\hline
		& 赤           & 緑           & 青           \\ \hline
		sample0 & 0bu2rh5g1aa & 0xox7ljdot4 & 0warg1fs2qj \\ \hline
		sample1 & 1ajauukb8mj & 1sjw40ypu0l & 1klos4n5aq6 \\ \hline
		sample2 & 2arwtbe1h2x & 2tndp3aqedx & 2ptohyekclg \\ \hline
		sample3 & 3apa5k0sxwa & 3viyx8b2ndm & 3nmrc7u0tb4 \\ \hline
		sample4 & 4deeqtvrucr & 4wf5i1dsksi & 4usyhe4fa31 \\ \hline
		sample5 & 5hjfteyeqim & 5sfi1bex4a3 & 5juyka6nm5g \\ \hline
		sample6 & 6aekcjml32b & 6i4d2i0q5le & 6b3e3up784p \\ \hline
		sample7 & 7cnk4rjdsuy & 7frelcmy6lh & 7e3qbglmuru \\ \hline
		sample8 & 8anykgc3hhr & 8y3mt7irhcr & 8mia1e7bbxr \\ \hline
		sample9 & 9sa1vx44ni4 & 9xiicrmnow8 & 9vjkso2sccd \\ \hline
	\end{tabular}
\end{table}


\begin{figure}[H]
	\centering
	\includegraphics[width=1.0\linewidth]{pics/experiment_situation.png}
	\caption{実験時の画像撮影環境}
	\label{fig:experiment_situation}
\end{figure}


撮影した画像を以下に示す.画像はそれぞれ,撮影距離 20cm, 撮影距離 40cm, 撮影距離 60cm, 撮影距離 80cmを表している.

\begin{figure}[H]
	\centering
	\includegraphics[width=1.0\linewidth]{pics/dist20.png}
	\caption{距離20 cmから撮影したときの画像}
	\label{fig:dist20}
\end{figure}

\begin{figure}[H]
	\centering
	\includegraphics[width=1.0\linewidth]{pics/dist40.png}
	\caption{距離40 cmから撮影したときの画像}
	\label{fig:dist40}
\end{figure}

\begin{figure}[H]
	\centering
	\includegraphics[width=1.0\linewidth]{pics/dist60.png}
	\caption{距離60 cmから撮影したときの画像}
	\label{fig:dist60}
\end{figure}

\begin{figure}[H]
	\centering
	\includegraphics[width=1.0\linewidth]{pics/dist80.png}
	\caption{距離80 cmから撮影したときの画像}
	\label{fig:dist80}
\end{figure}


比較対象として,従来のモノクロQRコードの撮影を行った.撮影された画像をまとめたものが図\ref{fig:dist_mono}である.

\begin{figure}[H]
	\centering
	\includegraphics[width=1.0\linewidth]{pics/dist_mono.png}
	\caption{比較対象とするモノクロQRコードの画像}
	\label{fig:dist_mono}
\end{figure}



\section{実験結果}
はじめに,カラーQRコードに対して,色抽出処理のみを行って,その検出率の比較を行った.
結果を表形式でまとめたものが表\ref{tab:color-extraction}である.



\begin{table}[H]   
    \centering
	\caption{色抽出処理のみを行った際の検出率の比較[\%]}
	\label{tab:color-extraction}
	\begin{tabular}{|l|l|l|l|l|}
		\hline
		& distance20cm & distance40cm & distance60cm & distance80cm \\ \hline
		Blue  & 100\%        & 20\%         & 0\%          & 0\%          \\ \hline
		Green & 100\%        & 0\%          & 0\%          & 0\%          \\ \hline
		Red   & 100\%        & 40\%         & 0\%          & 0\%          \\ \hline
	\end{tabular}
\end{table}

実験の結果を踏まえ,膨張処理を加えたうえで検出率の比較を行った.
膨張処理に適用するカーネルサイズはそれぞれ,$5\times 5$,$4\times 4$, $3\times 3$, $2\times 2$ を使用した.
各カーネルサイズでの検出率をまとめたものが表\ref{tab:kernel5}, \ref{tab:kernel4}, \ref{tab:kernel3}, 及び\ref{tab:kernel2}である.


\begin{table}[H]   
	\centering
	\caption{膨張処理時にカーネルサイズ$5\times 5$を適用した際の検出率[\%]}
	\label{tab:kernel5}
	\begin{tabular}{|l|l|l|l|l|}
		\hline
		& distance20cm & distance40cm & distance60cm & distance80cm \\ \hline
		Blue  & 100        & 0         & 0          & 0          \\ \hline
		Green & 100        & 0         & 0          & 0          \\ \hline
		Red   & 100        & 0         & 0          & 0          \\ \hline
	\end{tabular}
\end{table}

\begin{table}[H]   
	\centering
	\caption{膨張処理時にカーネルサイズ$4\times 4$を適用した際の検出率[\%]}
	\label{tab:kernel4}
	\begin{tabular}{|l|l|l|l|l|}
		\hline
		& distance20cm & distance40cm & distance60cm & distance80cm \\ \hline
		Blue  & 100        & 0         & 0          & 0          \\ \hline
		Green & 100        & 0         & 0          & 0          \\ \hline
		Red   & 100        & 0         & 0          & 0          \\ \hline
	\end{tabular}
\end{table}

\begin{table}[H]   
	\centering
	\caption{膨張処理時にカーネルサイズ$3\times3$を適用した際の検出率[\%]}
	\label{tab:kernel3}
	\begin{tabular}{|l|l|l|l|l|}
		\hline
		& distance20cm & distance40cm & distance60cm & distance80cm \\ \hline
		Blue  & 100        & 20       & 0          & 0          \\ \hline
		Green & 100        & 0        & 0          & 0          \\ \hline
		Red   & 100        & 40       & 0          & 0          \\ \hline
	\end{tabular}
\end{table}

\begin{table}[H]   
	\centering
	\caption{膨張処理時にカーネルサイズ$2\times 2$を適用した際の検出率[\%]}
	\label{tab:kernel2}
	\begin{tabular}{|l|l|l|l|l|}
		\hline
		& distance20cm & distance40cm & distance60cm & distance80cm \\ \hline
		Blue  & 100        & 20         & 0          & 0          \\ \hline
		Green & 100        & 0          & 0          & 0          \\ \hline
		Red   & 100        & 40         & 0          & 0          \\ \hline
	\end{tabular}
\end{table}


これに対し,従来のモノクロのQRコードは,距離80 cmからでも読み取りが可能であった.


\section{考察}

実験の結果として,膨張処理の追加による検出率の向上は見られなかった.
これは,画像処理の対象となるQRコード部分の解像度の低さが問題であると考えられる.
膨張処理前の画像と膨張処理後の画像は図\ref{fig:before_dilation}, 及び図\ref{fig:after_dilation}のとおりである.
モルフォロジー処理に使用したカーネルサイズが対象とするQRコード領域に対して大きかったことで,想定したような色抽出後の画像補正がなされなかったと考えられる.

\begin{figure}[H]
	\centering
	\includegraphics[width=1.0\linewidth]{pics/before_dilation.png}
	\caption{膨張処理前の画像}
	\label{fig:before_dilation}
\end{figure}


\begin{figure}[H]
	\centering
	\includegraphics[width=1.0\linewidth]{pics/after_dilation.png}
	\caption{膨張処理後の画像}
	\label{fig:after_dilation}
\end{figure}

また,一般に使用されているQRコードでは,同一サイズでも撮影距離80cmから検出できることから,本実験で使用したコード検出システムは既存のものに大きく劣ることが分かった.
