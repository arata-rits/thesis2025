%/**********************************************************************/
%		第4章:
%/**********************************************************************/
\chapter{検出性能の比較実験}
\label{lbl_chptr2}

本章では,従来のモノクロQRコードとカラーQRコードの検出率の比較実験について述べる.はじめに,4.1節で実験手法について述べ,4.2節で実験結果について述べる.また,4.3節では実験に対する考察を述べる.


%/**********************************************************************/
%		イントロの章
%/**********************************************************************/



\section{実験手法}
実験に際して.10種類のカラーQRコード画像を用意した.これは.カラーQRコードの種類による検出への影響を調べる為である.用意した10種類のカラーQRコードが図\ref{fig:CQR_sample0_9}である.

撮影距離について,20cm, 40cm, 60cm, 80cmの4種類の距離から撮影を行った.撮影時の実験環境を図\ref{fig:experiment_situation}に示す.



\begin{figure}[H]
	\centering
	\includegraphics[width=1.0\linewidth]{pics/CQR_sample0_9.png}
	\caption{実験に使用したカラーQRコード画像}
	\label{fig:CQR_sample0_9}
\end{figure}



\begin{figure}[H]
	\centering
	\includegraphics[width=1.0\linewidth]{pics/experiment_situation.png}
	\caption{実験時の画像の撮影環境}
	\label{fig:experiment_situation}
\end{figure}


撮影した画像を以下に示す.画像はそれぞれ,撮影距離20cm, 撮影距離40cm, 撮影距離60cm, 撮影距離80cmを表している.また,対照実験として,一般のQRコードの撮影も行った.


\begin{figure}[H]
	\centering
	\includegraphics[width=1.0\linewidth]{pics/dist20.png}
	\caption{20cmから撮影したときの画像}
	\label{fig:dist20}
\end{figure}

\begin{figure}[H]
	\centering
	\includegraphics[width=1.0\linewidth]{pics/dist40.png}
	\caption{40cmから撮影したときの画像}
	\label{fig:dist40}
\end{figure}

\begin{figure}[H]
	\centering
	\includegraphics[width=1.0\linewidth]{pics/dist60.png}
	\caption{60cmから撮影したときの画像}
	\label{fig:dist60}
\end{figure}

\begin{figure}[H]
	\centering
	\includegraphics[width=1.0\linewidth]{pics/dist80.png}
	\caption{80cmから撮影したときの画像}
	\label{fig:dist80}
\end{figure}


\section{実験結果}





\section{考察}
