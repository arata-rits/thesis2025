%/**********************************************************************/
%		第4章:
%/**********************************************************************/
\chapter{検出性能の比較実験}
\label{lbl_chptr2}

本章では,従来のモノクロQRコードとカラーQRコードの検出率の比較実験について述べる.はじめに,4.1節で実験手法について述べ,4.2節で実験結果について述べる.また,4.3節では実験に対する考察を述べる.


%/**********************************************************************/
%		イントロの章
%/**********************************************************************/



\section{実験手法}
実験に際して.10種類のカラーQRコード画像を用意した.これは.カラーQRコードの種類による検出への影響を調べる為である.用意した10種類のカラーQRコードが図\ref{fig:CQR_sample0_9}である.

撮影距離について,20cm, 40cm, 60cm, 80cmの4種類の距離から撮影を行った.撮影時の実験環境を図\ref{fig:experiment_situation}に示す.



\begin{figure}[H]
	\centering
	\includegraphics[width=1.0\linewidth]{pics/CQR_sample0_9.png}
	\caption{実験に使用したカラーQRコード画像}
	\label{fig:CQR_sample0_9}
\end{figure}



\begin{figure}[H]
	\centering
	\includegraphics[width=1.0\linewidth]{pics/experiment_situation.png}
	\caption{実験時の画像の撮影環境}
	\label{fig:experiment_situation}
\end{figure}


撮影した画像を以下に示す.画像はそれぞれ,撮影距離20cm, 撮影距離40cm, 撮影距離60cm, 撮影距離80cmを表している.また,対照実験として,一般のQRコードの撮影も行った.


\begin{figure}[H]
	\centering
	\includegraphics[width=1.0\linewidth]{pics/dist20.png}
	\caption{20cmから撮影したときの画像}
	\label{fig:dist20}
\end{figure}

\begin{figure}[H]
	\centering
	\includegraphics[width=1.0\linewidth]{pics/dist40.png}
	\caption{40cmから撮影したときの画像}
	\label{fig:dist40}
\end{figure}

\begin{figure}[H]
	\centering
	\includegraphics[width=1.0\linewidth]{pics/dist60.png}
	\caption{60cmから撮影したときの画像}
	\label{fig:dist60}
\end{figure}

\begin{figure}[H]
	\centering
	\includegraphics[width=1.0\linewidth]{pics/dist80.png}
	\caption{80cmから撮影したときの画像}
	\label{fig:dist80}
\end{figure}


比較対象として,従来のモノクロQRコードの撮影を行った.撮影された画像をまとめたものが図\ref{fig:dist_mono}である.

\begin{figure}[H]
	\centering
	\includegraphics[width=1.0\linewidth]{pics/dist_mono.png}
	\caption{比較対象とするモノクロQRコードの画像}
	\label{fig:dist_mono}
\end{figure}



\section{実験結果}
はじめに,カラーQRコードに対して,色抽出処理のみを行って,その検出率の比較を行った.
結果を表形式でまとめたものが表\ref{tab:color-extraction}である.



\begin{table}[H]   
    \centering
	\caption{色抽出処理のみを行った際の検出率の比較}
	\label{tab:color-extraction}
	\begin{tabular}{|l|l|l|l|l|}
		\hline
		& distance20cm & distance40cm & distance60cm & distance80cm \\ \hline
		Blue  & 100\%        & 20\%         & 0\%          & 0\%          \\ \hline
		Green & 100\%        & 0\%          & 0\%          & 0\%          \\ \hline
		Red   & 100\%        & 40\%         & 0\%          & 0\%          \\ \hline
	\end{tabular}
\end{table}

実験の結果を踏まえ,膨張処理を加えたうえで検出率の比較を行った.
膨張処理に適用するカーネルサイズはそれぞれ,$5\times 5$,$4\times 4$, $3\times 3$, $2\times 2$ を使用した.
各カーネルサイズでの検出率をまとめたものが表\ref{tab:kernel5}, 表\ref{tab:kernel4}, 表\ref{tab:kernel3}, 表\ref{tab:kernel2}である.


\begin{table}[H]   
	\centering
	\caption{膨張処理時にカーネルサイズ$5\times 5$を適用した際の検出率}
	\label{tab:kernel5}
	\begin{tabular}{|l|l|l|l|l|}
		\hline
		& distance20cm & distance40cm & distance60cm & distance80cm \\ \hline
		Blue  & 100\%        & 0\%         & 0\%          & 0\%          \\ \hline
		Green & 100\%        & 0\%         & 0\%          & 0\%          \\ \hline
		Red   & 100\%        & 0\%         & 0\%          & 0\%          \\ \hline
	\end{tabular}
\end{table}

\begin{table}[H]   
	\centering
	\caption{膨張処理時にカーネルサイズ$4\times 4$を適用した際の検出率}
	\label{tab:kernel4}
	\begin{tabular}{|l|l|l|l|l|}
		\hline
		& distance20cm & distance40cm & distance60cm & distance80cm \\ \hline
		Blue  & 100\%        & 0\%         & 0\%          & 0\%          \\ \hline
		Green & 100\%        & 0\%         & 0\%          & 0\%          \\ \hline
		Red   & 100\%        & 0\%         & 0\%          & 0\%          \\ \hline
	\end{tabular}
\end{table}

\begin{table}[H]   
	\centering
	\caption{膨張処理時にカーネルサイズ$3\times3$を適用した際の検出率}
	\label{tab:kernel3}
	\begin{tabular}{|l|l|l|l|l|}
		\hline
		& distance20cm & distance40cm & distance60cm & distance80cm \\ \hline
		Blue  & 100\%        & 20\%         & 0\%          & 0\%          \\ \hline
		Green & 100\%        & 0\%          & 0\%          & 0\%          \\ \hline
		Red   & 100\%        & 40\%         & 0\%          & 0\%          \\ \hline
	\end{tabular}
\end{table}

\begin{table}[H]   
	\centering
	\caption{膨張処理時にカーネルサイズ$2\times 2$を適用した際の検出率}
	\label{tab:kernel2}
	\begin{tabular}{|l|l|l|l|l|}
		\hline
		& distance20cm & distance40cm & distance60cm & distance80cm \\ \hline
		Blue  & 100\%        & 20\%         & 0\%          & 0\%          \\ \hline
		Green & 100\%        & 0\%          & 0\%          & 0\%          \\ \hline
		Red   & 100\%        & 40\%         & 0\%          & 0\%          \\ \hline
	\end{tabular}
\end{table}



\section{考察}

実験の結果として,膨張処理の追加による検出率の向上は見られなかった.
これは,画像処理の対象となるQRコード部分の解像度の低さが問題であると考えられる.
膨張処理前の画像と膨張処理後の画像は下図のとおりである(図\ref{before_dilation}), (図\ref{after_dilation}).
	

\begin{figure}[H]
	\centering
	\includegraphics[width=1.0\linewidth]{pics/before_dilation.png}
	\caption{膨張処理前の画像}
	\label{fig:before_dilation}
\end{figure}


\begin{figure}[H]
	\centering
	\includegraphics[width=1.0\linewidth]{pics/after_dilation.png}
	\caption{膨張処理後の画像}
	\label{fig:after_dilation}
\end{figure}
