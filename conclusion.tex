%/**********************************************************************/
%		結論:
%/**********************************************************************/
\chapter{結論}
\label{lbl_chptr2}

本章では,本研究のまとめと今後の展望について述べる


%/**********************************************************************/
%		イントロの章
%/**********************************************************************/



\section{まとめ}
本研究を通じて,RGBを重ね合わせる事によってエンコードされるカラーQRコードの実装と,近距離からの撮影によるデコードが可能となるシステムを構築することができた.また,カラーQRコードを入力画像としたアートQRコードの生成に関して,今回の実験で扱った画像生成手法では,開発した読み取りシステムでデコード可能な画像の生成はできなかった.

\section{今後の展望}

\subsection{カラーQRコードの読み取りシステムについて}
本実験により,中長距離からの撮影で,カラーコードが認識できないという課題が見つかった.この課題に対する展望として,物体検出AI等を利用した画像の補正が挙げられる.本研究で行った実験では,撮影対象に対する距離に関わらず,同様のカーネルサイズを用いて,撮影画像に対する前処理を行っていた.これにより,撮影距離が離れるにつれて,撮影画像中のQRコード領域が小さくなり,処理後の画像に劣化が見られた.そこで,物体検出アルゴリズムを用いて,先にQRコード領域の推定を行い,補正画像を作成したうえで,モルフォロジー処理を適用することでカラーQRコードの検出率向上が期待される(図\ref{fig:matome_1}). このような処理は,カラーQRコードだけではなく,一般のQRコードの検出率を向上にも役立つものであるだろう.


\begin{figure}[H]
	\centering
	\includegraphics[width=0.8\linewidth]{pics/matome_1.png}
	\caption{カラーQRコードの検出率を向上させる改善案}
	\label{fig:matome_1}
\end{figure}




\subsection{カラーQRコード入力によるアートQRコードの生成について}
本研究で取り扱った手法では,カラーQRコードの読み取りシステムで検出可能なカラーアートQRコード画像を生成することができなかった.しかし,出力画像に大きく影響するという理由からcontrolnetに対する工夫は必要不可欠でる.本技術に対する展望としては,controlnetモデルの独自開発が挙げられる.インターネット上に公開されているcontrolnetモデルの多くは,safetensorと呼ばれるファイル形式をとっており,内部はバイナリのファイルとなっている.従って,ダウンロードしたモデルの解析や改変は現実的ではない.そこで開発工数は多くなるが,モデルの独自開発が必要とされると考えられる.