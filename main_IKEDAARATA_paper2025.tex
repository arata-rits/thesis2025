\documentclass[12pt,eclepsfx,a4j,twoside,openright]{jreport}
\usepackage[dvipdfmx]{graphicx}
\usepackage{./tex_files/hangcaption}
\usepackage{amssymb}
\usepackage{./tex_files/fancyheadings}
\usepackage{pifont}
\usepackage{url}
%\usepackage{amsmath}
\usepackage{./tex_files/cite}
\renewcommand\citeleft{}
\renewcommand\citeright{}
\renewcommand\citeform[1]{[#1]}
\usepackage{./tex_files/here}
\usepackage[subrefformat=parens]{subcaption}
%\usepackage{listings, jlisting}

\usepackage{hyperref}
\usepackage{listings}
\usepackage{enumitem}
\usepackage{rotating}



\lstset{
	basicstyle=\ttfamily\small,
	numbers=left,
	numberstyle=\tiny,
	frame=single,
	breaklines=true,
	tabsize=2,
	captionpos=b
}


\begin{document}
\clearpage
\thispagestyle{empty}
\vspace*{20pt}
%\newpage
\thispagestyle{empty}

%/**********************************************************************/
%		表紙
%/**********************************************************************/

\thispagestyle{empty}
\vspace*{5mm}
\begin{center}
{\LARGE 修士論文 \\}	%卒業論文,修士論文,博士論文のいずれか
\vspace{15mm}
\baselineskip=13mm
{\Huge{カラーQRコードの実装とその応用に関する研究}}	%論文のタイトル,改行したい場合は\\で区切る
\vspace{7mm}

%{\Large
%\[ \mbox{Research on performance evaluation of}\choose
%\mbox{SoC installed in recent smartphones}
%\]}	%英語のタイトルをつけるとかっこいい

{\Large
\[ \mbox{Implementation of color QR codes and a study on detectable}\choose
   \mbox {illustration generation methods using image generation AI}
\]}	%英語のタイトルをつけるとかっこいい



\vspace{45mm}
%{\LARGE 2006年\\}
{\Huge{池田 新}\\}		%あなたの名前



\vspace{10mm}
{\LARGE 立命館大学大学院\\理工学研究科電子システム専攻\\		%所属です,学部と研究科は所属が違うので注意
}
\vspace{10mm}
{\LARGE 2026年3月\\}		%日付です
\end{center}

\newpage
\thispagestyle{empty}
 
 
\newpage
%/**********************************************************************/
%		概要
%/**********************************************************************/
\renewcommand{\abstractname}{内容梗概}

\pagenumbering{roman}
%\baselineskip=32pt
\begin{abstract}
\thispagestyle{plain}


近年,技術的背景や,権利的背景,社会的背景から, QRコード(Quick Response code)が広く普及している.QRコードは,高速読み取りが可能な二次元コードの一種であり,従来使用されてきたバーコードに比べて,記録可能な情報量が多く,誤り訂正能力が強いかつ高速な読み取りが可能という技術的な優位性があった.また,自動車部品のトレーサビリティ管理のために開発されたという背景がありながらも,特許権を行使しないと宣言されたことで世界中での特許フリーな使用が可能となっている.さらには,スマートフォンのような,高性能かつ多機能なデバイスが世界的に普及したことで,広告媒体や,決済手法としてQRコードの利用がなされている.しかし,現状のQRコードは単一の色の濃淡で表現されており,その色については,符号化する情報量増加の余地が残されている.
そこで本研究では,RGBを用いたカラーQRコードシステムの実装を行い,その有用性について検討した.本研究で扱うカラーQRコードは,RGBの単色カラー画像に変換したQRコードを重ね合わせることにより作成し,計8色(赤,青,緑.シアン,マゼンタ,イエロー,白,黒)のカラー画像となっている.具体的には,カラーQRコードの検出アプリの実装,従来のQRとの検出距離の比較実験を行った.また,近年技術向上が進んでいる画像生成AIを使用したQRコードの検出が可能なイラスト生成AIの実装方法について検討を行った.
本研究では、撮影距離約20 cmの近距離において安定した検出が可能な読み取りアプリを実装した。さらに、カラーQRコードを入力画像としたQRコード検出可能な画像の生成を目的として,画像の生成環境を構築し、既存のControlNetを用いた画像生成を通して、生成手法の検討を行った。
 
\end{abstract}


%========================================
%========================================
%    論文本体
%========================================
%========================================
%\setcounter{tocdepth}{3}
\pagestyle{fancy}
\setcounter{page}{3}
%\pagenumbering{roman}
\tableofcontents
%\clearpage
\listoffigures
%\clearpage
\listoftables
\clearpage

\lhead{第 \thechapter 章} \rhead{}		%ここから各章毎のファイルを作成して論文とします.
%/**********************************************************************/
%		第1章:序論
%/**********************************************************************/
\pagenumbering{arabic}
\chapter{序論}
\label{lbl_chptr1}

%/**********************************************************************/
%		研究背景
%/**********************************************************************/

\section{研究背景}
\label{lbl_cp1_haikei}

近年,高速かつ安定した読み取りが可能な2次元コードであるQRコード(Quick Response code)が世界的に使用されている.このような技術の普及には,大きく分けて三つの背景があると考えられる.それは,技術的背景,権利的背景,社会的背景の3つである.

初めに,技術的背景について,これはQRコードが持つ技術的な優位性から由来する.従来使用されてきたバーコードと比較して,QRコードは記録可能な情報量が多く,誤り訂正能力が強い,かつ高速で安定した読み取りが可能というように使用するメリットが複数存在している.

次に,権利的な背景について,これはQRコードが特許フリーなため誰でも自由に使用可能なことに由来する.QRコードは自動車部品のトレーサビリティ管理のためにデンソーの一事業部(現デンソーウェーブ)によって開発された.しかし,デンソーはその仕様をオープンにし,特許の自由な使用を許可した.これにより,QRコード決済や,航空券等のチケット管理など,本来開発者が想定していなかったような用途でも使われ,社会に広く浸透していった.
 
 最後に社会的な背景について,これはQRコードの読み取りを可能にする電子機器,特にスマートフォンの普及に由来する.半導体製造技術の向上により,高性能かつ低消費電力な電子デバイスが安価に手に入るようになったことで,2010年代からスマートフォンが急速に普及してきた.日本において,その保有率は2010年で10%程度に留まっていたものが,2021年には90%近くに達している\cite{spread-smartphones}.

\begin{figure}[H]
	\centering
	\includegraphics[width=1.0\linewidth]{pics/SmartPhone_popularization.png}
	\caption{電子機器の普及率の推移}
	\label{fig:SmartPhone_popularization}
\end{figure}

このスマートフォンの普及は,誰もがQRコードを読み取り,情報を取得する事ができるという土台を作り,QRコードの普及を後押ししただろう.現にSNSアカウントの共有や,市中の広告でもQRコードが使用されている.


\section{本研究の目的}
本研究の目的は大きく二つに分けられる.一つ目は,面積当たりの情報量を増やすこと.二つ目は.広告媒体等で他のデザインを既存しないような二次元コードを作成することである.











%/**********************************************************************/
%		第2章:
%/**********************************************************************/
\chapter{カラーQRコードに関する先行研究}
\label{lbl_chptr2}

アブスト


%/**********************************************************************/
%		イントロの章
%/**********************************************************************/



\section{セクション1}

\section{セクション2}

\section{セクション3}

%/**********************************************************************/
%		第4章:
%/**********************************************************************/
\chapter{カラーQRコードの読み取りシステムの開発}
\label{lbl_chptr2}

アブスト


%/**********************************************************************/
%		イントロの章
%/**********************************************************************/



\section{エンコード方法}

\section{デコード方法}

\section{読み取りアプリの開発}

%/**********************************************************************/
%		第4章:
%/**********************************************************************/
\chapter{検出性能の比較実験}
\label{lbl_chptr2}

本章では,従来のモノクロQRコードとカラーQRコードの検出率の比較実験について述べる.はじめに,4.1節で実験手法について述べ,4.2節で実験結果について述べる.また,4.3節では実験に対する考察を述べる.


%/**********************************************************************/
%		イントロの章
%/**********************************************************************/



\section{実験手法}
実験に際して.10種類のカラーQRコード画像を用意した.これは.カラーQRコードの種類による検出への影響を調べる為である.用意した10種類のカラーQRコードが図\ref{fig:CQR_sample0_9}である.

撮影距離について,20cm, 40cm, 60cm, 80cmの4種類の距離から撮影を行った.撮影時の実験環境を図\ref{fig:experiment_situation}に示す.



\begin{figure}[H]
	\centering
	\includegraphics[width=1.0\linewidth]{pics/CQR_sample0_9.png}
	\caption{実験に使用したカラーQRコード画像}
	\label{fig:CQR_sample0_9}
\end{figure}



\begin{figure}[H]
	\centering
	\includegraphics[width=1.0\linewidth]{pics/experiment_situation.png}
	\caption{実験時の画像の撮影環境}
	\label{fig:experiment_situation}
\end{figure}


撮影した画像を以下に示す.画像はそれぞれ,撮影距離20cm, 撮影距離40cm, 撮影距離60cm, 撮影距離80cmを表している.また,対照実験として,一般のQRコードの撮影も行った.


\begin{figure}[H]
	\centering
	\includegraphics[width=1.0\linewidth]{pics/dist20.png}
	\caption{20cmから撮影したときの画像}
	\label{fig:dist20}
\end{figure}

\begin{figure}[H]
	\centering
	\includegraphics[width=1.0\linewidth]{pics/dist40.png}
	\caption{40cmから撮影したときの画像}
	\label{fig:dist40}
\end{figure}

\begin{figure}[H]
	\centering
	\includegraphics[width=1.0\linewidth]{pics/dist60.png}
	\caption{60cmから撮影したときの画像}
	\label{fig:dist60}
\end{figure}

\begin{figure}[H]
	\centering
	\includegraphics[width=1.0\linewidth]{pics/dist80.png}
	\caption{80cmから撮影したときの画像}
	\label{fig:dist80}
\end{figure}


\section{実験結果}





\section{考察}

%/**********************************************************************/
%		第5章:
%/**********************************************************************/
\chapter{カラーQRコードを用いたAIイラストの生成}
\label{lbl_chptr2}

アブスト


%/**********************************************************************/
%		イントロの章
%/**********************************************************************/



\section{イラスト生成AIの原理}

\section{使用したソフトとモデル}

\section{実験結果}

\section{考察}

%\input{chptr6}
%%/**********************************************************************/
%		第7章:
%/**********************************************************************/
\chapter{タイトル}
\label{lbl_chptr2}

アブスト


%/**********************************************************************/
%		イントロの章
%/**********************************************************************/



\section{セクション1}

\section{セクション2}

\section{セクション3}

%/**********************************************************************/
%		結論:
%/**********************************************************************/
\chapter{結論}
\label{lbl_chptr2}

本章では,本研究のまとめと今後の展望について述べる


%/**********************************************************************/
%		イントロの章
%/**********************************************************************/



\section{まとめ}
本研究を通じて,RGBを重ね合わせる事によってエンコードされるカラーQRコードの実装と,近距離からの撮影によるデコードが可能となるシステムを構築することができた.また,カラーQRコードを入力画像としたアートQRコードの生成に関して,今回の実験で扱った画像生成手法では,開発した読み取りシステムでデコード可能な画像の生成はできなかった.

\section{今後の展望}

\subsection{カラーQRコードの読み取りシステムについて}
本実験により,中長距離からの撮影で,カラーコードが認識できないという課題が見つかった.この課題に対する展望として,物体検出AI等を利用した画像の補正が挙げられる.本研究で行った実験では,撮影対象に対する距離に関わらず,同様のカーネルサイズを用いて,撮影画像に対する前処理を行っていた.これにより,撮影距離が離れるにつれて,撮影画像中のQRコード領域が小さくなり,処理後の画像に劣化が見られた.そこで,物体検出アルゴリズムを用いて,先にQRコード領域の推定を行い,補正画像を作成したうえで,モルフォロジー処理を適用することでカラーQRコードの検出率向上が期待される(図\ref{fig:matome_1}). このような処理は,カラーQRコードだけではなく,一般のQRコードの検出率を向上にも役立つものであるだろう.


\begin{figure}[H]
	\centering
	\includegraphics[width=0.8\linewidth]{pics/matome_1.png}
	\caption{カラーQRコードの検出率を向上させる改善案}
	\label{fig:matome_1}
\end{figure}




\subsection{カラーQRコード入力によるアートQRコードの生成について}
本研究で取り扱った手法では,カラーQRコードの読み取りシステムで検出可能なカラーアートQRコード画像を生成することができなかった.しかし,出力画像に大きく影響するという理由からcontrolnetに対する工夫は必要不可欠でる.本技術に対する展望としては,controlnetモデルの独自開発が挙げられる.インターネット上に公開されているcontrolnetモデルの多くは,safetensorと呼ばれるファイル形式をとっており,内部はバイナリのファイルとなっている.従って,ダウンロードしたモデルの解析や改変は現実的ではない.そこで開発工数は多くなるが,モデルの独自開発が必要とされると考えられる.





%------------------------------------------------------------------
% 参考文献
%------------------------------------------------------------------

\addcontentsline{toc}{chapter}{参考文献}%参考文献は大変重要です,bibtexというものを使用すると使い回しができて便利です.
\bibliographystyle{./tex_files/ieice_e}
\bibliography{./tex_files/e_fmcam_ref}		%これがbibtexファイルです.参考文献のリストです.

%------------------------------------------------------------------
%付録
%------------------------------------------------------------------

\appendix
\lhead{付録} \rhead{}
\appendix
\chapter*{付録}
\label{huroku}
\lstset{language={c}, basicstyle={\tiny}}

\section*{クロスフリッカ式ステゴパネル制御プログラム}
\label{clossflicker}


\subsection*{sutego3.c}
\label{huroku_1}
\lstset{language={c}, basicstyle={\tiny}}
%\lstinputlisting[label=sutego3]{sutego3.c}
\textbf{}


\section*{2段ダイナミック式ステゴパネル制御プログラム}
\label{2dinamic}


\subsection*{sutegoR1219.c}
\label{huroku_2}
\lstset{language={c}, basicstyle={\tiny}}
\%lstinputlisting[label=sutegoR1219]{sutegoR1219.c}
\textbf{}



\section*{トリケラパネル制御プログラム}
\label{tricolor1127}



\subsection*{tricolor1127.c}
\label{huroku_3}
\lstset{language={c}, basicstyle={\tiny}}
%\lstinputlisting[label=tricolor1127]{tricolor1127.c}
\textbf{}



\newpage

%------------------------------------------------------------------
\chapter*{謝辞}							%謝辞は必ず書きます.好きにかけるのでどんどん書きましょう.
\lhead{謝辞} \rhead{}
\addcontentsline{toc}{chapter}{謝辞}
%------------------------------------------------------------------
本論文の作成にあたり, 貴重な助言, ご指導をして頂いた立命館大学理工学部電子情報工学科 熊木 武志教授に深く感謝の意を表します. また, 本研究に関わりご助言をして頂いた石田勝之介氏に深く感謝致します. 最後に, 日頃から様々な事においてお世話になりました7期生を始めとするマルチメディア集積回路システム研究室の皆様に最大の感謝をお贈り致します.

\vspace{15mm}
\begin{flushright}
2026年3月 池田 新
\end{flushright}
%------------------------------------------------------------------
%------------------------------------------------------------------
\chapter*{研究業績リスト}	
\addcontentsline{toc}{chapter}{発表論文リスト}
\lhead{発表論文リスト} \rhead{}
%------------------------------------------------------------------
%皆さんの研究業績をここにどんどん書いてきましょう!筆頭でも共著でも良いです.少しでも良いので書くとかっこいいです.修士の場合は論文と国際学会は欲しいですね.

\small
\noindent
{\bf 【国内研究会等発表】}		%ここに展示会参加等を書きましょう
\begin{itemize}

	\item
	\underline{池田新}, 熊木武志, "RGBカラーQRコードの読み取り実験とその評価について,"\\
	ソサイエティ大会 2024, Sep., 2024.
	
	
\end{itemize}
{\bf 【国外研究会等発表】}
\begin{itemize}
	\item
	\underline{Arata Ikeda}, Takumi Hayashi, and Takeshi Kumaki, "Development of secure QR code by using invisible information display lighting device,"
	ITC-CSCC 2024, Jul., 2024
	
	\item
	\underline{Arata Ikeda}, Tomoki Yamashita, and Takeshi Kumaki, "Development of RGB micro QR code system using invisible information lighting device 'Stego-panel V',"
	NOLTA 2025, Oct., 2025

\end{itemize}
{\bf 【その他研究活動】}		%ここに展示会参加等を書きましょう
\begin{itemize}
	\item 
	EdgeTech+ West 2024, Jul., 2024
	\item 
	EdgeTech+ 2024, Nov., 2024
	\item 
	EdgeTech+ West 2025, Jul., 2025
	\item 
	EdgeTech+ 2025, Nov., 2025
	
	

\end{itemize}
{\bf 【受賞】}		%ここに展示会参加等を書きましょう

\clearpage

\end{document}