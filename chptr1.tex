%/**********************************************************************/
%		第1章:序論
%/**********************************************************************/
\pagenumbering{arabic}
\chapter{序論}
\label{lbl_chptr1}

%/**********************************************************************/
%		研究背景
%/**********************************************************************/

\section{研究背景}
\label{lbl_cp1_haikei}

近年,高速かつ安定した読み取りが可能な2次元コードであるQRコード(Quick Response code)が世界的に使用されている.このような技術の普及には,大きく分けて三つの背景があると考えられる.それは,技術的背景,権利的背景,社会的背景の3つである.

初めに,技術的背景について,これはQRコードが持つ技術的な優位性から由来する.従来使用されてきたバーコードと比較して,QRコードは記録可能な情報量が多く,誤り訂正能力が強い,かつ高速で安定した読み取りが可能というように使用するメリットが複数存在している.

次に,権利的な背景について,これはQRコードが特許フリーなため誰でも自由に使用可能なことに由来する.QRコードは自動車部品のトレーサビリティ管理のためにデンソーの一事業部(現デンソーウェーブ)によって開発された.しかし,デンソーはその仕様をオープンにし,特許の自由な使用を許可した.これにより,QRコード決済や,航空券等のチケット管理など,本来開発者が想定していなかったような用途でも使われ,社会に広く浸透していった.
 
 最後に社会的な背景について,これはQRコードの読み取りを可能にする電子機器,特にスマートフォンの普及に由来する.半導体製造技術の向上により,高性能かつ低消費電力な電子デバイスが安価に手に入るようになったことで,2010年代からスマートフォンが急速に普及してきた.日本において,その保有率は2010年で10%程度に留まっていたものが,2021年には90%近くに達している\cite{spread-smartphones}.\footnote{総務省 令和4年版 情報通信白書 https://www.soumu.go.jp/johotsusintokei/whitepaper/ja/r04/html/nd238110.htmlより引用}

\begin{figure}[H]
	\centering
	\includegraphics[width=1.0\linewidth]{pics/SmartPhone_popularization.png}
	\caption{電子機器の普及率の推移}
	\label{fig:SmartPhone_popularization}
\end{figure}

このスマートフォンの普及は,誰もがQRコードを読み取り,情報を取得する事ができるという土台を作り,QRコードの普及を後押ししただろう.現にSNSアカウントの共有や,市中の広告でもQRコードが使用されている.

\section{QRコードについて}
QRコード(Quick Response code)は,1994年に株式会社デンソーの一事業部(現株式会社デンソーウェーブ)によって開発された,高速かつ安定した読み取りが可能なマトリクス型の2次元コードの一種である.本コードの特徴的な形状は囲碁が元となっている.同様に広く普及しているバーコードと比較して,縦と横に情報を格納していることから,より大容量のデータをエンコードすることができ,省スペースに印字をすることができる.QRコードの構造について,コードを構成する白黒の正方形はセルと呼ばれる.このセルの組み合わせにより情報が表され,QRコード内部には,データセル,切り出しシンボル,タイミングパターン,アライメントパターン,フォーマット情報が存在している(図\ref{fig:KDDI_qr_structure}).\footnote{QR コードってどういう仕組み?種類や歴史、使用時の注意点などを解説|KDDI トビラ(https://time-space.kddi.com/ict-keywords/20190425/2624)より引用}



\begin{figure}[H]
	\centering
	\includegraphics[width=1.0\linewidth]{pics/KDDI_qr_structure.png}
	\caption{QRコードの構造}
	\label{fig:KDDI_qr_structure}
\end{figure}

QRコードは縦横を構成するセル数毎にバージョンが決められている(図\ref{fig:qr_version}).バージョンは1から40まで設定されており,例えば,バージョン1では$21 \times 21$セル, バージョン40では$177 \times 177$セルとなっている.最小サイズであるバージョン1では,数字で17文字,英数字で10文字,漢字で4文字,バイナリで7ビットのデータ表現が可能である.また,QRコードには3つの角に配置されている特徴的なパターンがある.このパターンをファインダパターンもしくは位置検出パターンと呼ぶ.デコード時にはこのパターンを利用することで、QRコードの位置特定を行う。具体的には、3つの位置検出パターンを基準として、QRコードの向きや回転状態を検出し、正しい方向に補正する。次に、位置検出パターン間の距離を測定することでコードのサイズを認識し、3つのパターンの相対的な配置を基に、QRコード全体の四角形領域を特定する。

図\ref{}に示すように、A、B、Cのいずれの位置においても、白黒部分の幅の比率は1:1:3:1:1となっている。この比率はQRコードが回転した場合でも保持されるため、位置検出パターンの検出結果およびそれらの位置関係から回転角度を認識することができる。その結果、360°いずれの方向からでも読み取りが可能となり、処理の効率化が実現されている。

\begin{figure}[H]
	\centering
	\includegraphics[width=1.0\linewidth]{pics/qr_version.png}
	\caption{QRコードのバージョン}
	\label{fig:qr_version}
\end{figure}



\begin{figure}[H]
	\centering
	\includegraphics[width=1.0\linewidth]{pics/finder_pattern.png}
	\caption{ファインダパターンの原理}
	\label{fig:finder_pattern}
\end{figure}



\section{本研究の目的}
本研究の目的は大きく二つに分けられる.一つ目は,面積当たりの情報量を増やすこと.二つ目は.広告媒体等で他のデザインを既存しないような二次元コードを作成することである.










