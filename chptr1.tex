%/**********************************************************************/
%		第1章:序論
%/**********************************************************************/
\pagenumbering{arabic}
\chapter{序論}
\label{lbl_chptr1}

%/**********************************************************************/
%		研究背景
%/**********************************************************************/

\section{研究背景}
\label{lbl_cp1_haikei}

近年,高速かつ安定した読み取りが可能な2次元コードであるQRコード(Quick Response code)が世界的に使用されている.このような技術の普及には,大きく分けて三つの背景があると考えられる.それは,技術的背景,権利的背景,社会的背景の3つである.

初めに,技術的背景について,これはQRコードが持つ技術的な優位性から由来する.従来使用されてきたバーコードと比較して,QRコードは記録可能な情報量が多く,誤り訂正能力が強い,かつ高速で安定した読み取りが可能というように使用するメリットが複数存在している.

次に,権利的な背景について,これはQRコードが特許フリーなため誰でも自由に使用可能なことに由来する.QRコードは自動車部品のトレーサビリティ管理のためにデンソーの一事業部(現デンソーウェーブ)によって開発された.しかし,デンソーはその仕様をオープンにし,特許の自由な使用を許可した.これにより,QRコード決済や,航空券等のチケット管理など,本来開発者が想定していなかったような用途でも使われ,社会に広く浸透していった.
 
 最後に社会的な背景について,これはQRコードの読み取りを可能にする電子機器,特にスマートフォンの普及に由来する.半導体製造技術の向上により,高性能かつ低消費電力な電子デバイスが安価に手に入るようになったことで,2010年代からスマートフォンが急速に普及してきた.日本において,その保有率は2010年で10%程度に留まっていたものが,2021年には90%近くに達している\cite{spread-smartphones}.

\begin{figure}[H]
	\centering
	\includegraphics[width=1.0\linewidth]{pics/SmartPhone_popularization.png}
	\caption{電子機器の普及率の推移}
	\label{fig:SmartPhone_popularization}
\end{figure}

このスマートフォンの普及は,誰もがQRコードを読み取り,情報を取得する事ができるという土台を作り,QRコードの普及を後押ししただろう.現にSNSアカウントの共有や,市中の広告でもQRコードが使用されている.


\section{本研究の目的}
本研究の目的は大きく二つに分けられる.一つ目は,面積当たりの情報量を増やすこと.二つ目は.広告媒体等で他のデザインを既存しないような二次元コードを作成することである.










