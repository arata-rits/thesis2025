%/**********************************************************************/
%		第1章:序論
%/**********************************************************************/
\pagenumbering{arabic}
\chapter{序論}
\label{lbl_chptr1}

%/**********************************************************************/
%		研究背景
%/**********************************************************************/

\section{研究背景}
\label{lbl_cp1_haikei}

近年,高速,かつ安定した読み取りが可能な2次元コードであるQRコード(Quick Response code)が世界的に使用されている.この技術が普及した理由として,大きく分けて三つの背景がある.それは,技術的,権利的,そして社会的背景である.

始めに,技術的背景について述べる.これはQRコードが持つ技術的な優位性に由来するものであり,従来使用されてきた1次元バーコードと比較して,QRコードは2次元で構成されており,記録可能な情報量が多く,誤り訂正能力が強い,かつ高速で安定した読み取りが可能等,利点が複数存在している.

次に,権利的な背景について述べる.QRコードは1994年に特許出願\cite{qr_jp}がなされたにもかかわらず,誰でも自由に使用可能である.QRコードは自動車部品のトレーサビリティ管理のためにデンソーの一事業部(現デンソーウェーブ)によって開発された\cite{QR_dev}.しかしながら,デンソーはその仕様をオープンにし,特許の自由な使用を許可した.これにより,QRコード決済や,航空券等のチケット管理など,本来開発者が想定していなかったような用途でも使われ,社会に広く浸透していった. ここで,QRコードの普及率を示す例として,世界各国におけるQRコード決済サービスの普及率を示す(図\ref{fig:QR_diffusion_rate}).総務省の2021年の調査によると,日本におけるQRコード決済サービスの普及率は既に5割を超えており,更には,日本よりQRコード決済が普及している中国では,その普及率が66\%にまで達している.

\begin{figure}[H]
	\centering
	\includegraphics[width=0.8\linewidth]{pics/QR_diffusion_rate.png}
	\caption[QRコード決済サービスの普及率(各国比較)]{QRコード決済サービスの普及率(各国比較)\protect\footnotemark}
	\label{fig:QR_diffusion_rate}
\end{figure}

\footnotetext{総務省(2021)「ウィズコロナにおけるデジタル活用の実態と利用者意識の変化に関する\\調査研究」\url{(https://www.soumu.go.jp/johotsusintokei/linkdata/r03_01_houkoku.pdf)}\\を元に作成}

最後に社会的な背景について述べる.これはQRコードの読み取りを可能にする電子機器,特にスマートフォンの普及に由来する.半導体製造技術の向上により,高性能,かつ低消費電力な電子デバイスが安価に手に入るようになったことで,2010年代からスマートフォンが急速に普及してきた.図\ref{fig:SmartPhone_popularization}は,日本における電子機器の普及率推移を示したものである.ここで,グラフの横軸は,調査年を示し,縦軸は普及率(\%)を示している.総務省の情報通信白書によると,日本においてスマートフォンの保有率は2010年で10%程度に留まっていたものが,2021年には90%近くに達している\cite{spread-smartphones}.\footnote{総務省 令和4年版 情報通信白書(\url{https://www.soumu.go.jp/johotsusintokei/whitepaper/ja/r04/html/nd238110.html})より引用} このスマートフォンの普及は,誰もがQRコードを読み取り,情報を取得する事ができるという土台を作り,QRコードの普及を強く後押しした.

\begin{figure}[H]
	\centering
	\includegraphics[width=1.0\linewidth]{pics/SmartPhone_popularization.png}
	\caption{電子機器の普及率の推移}
	\label{fig:SmartPhone_popularization}
\end{figure}



\section{QRコードについて}
QRコード(Quick Response code)は,1994年に株式会社デンソーの一事業部(現株式会社デンソーウェーブ)によって開発された,高速,かつ安定した読み取りが可能なマトリクス型の2次元コードの一種である\cite{QR_denso}.本コードの特徴的な形状は囲碁が元となっている.同様に広く普及している1次元バーコードと比較して,縦と横の2次元で情報を格納していることから,より大容量のデータをエンコードすることができ,なおかつ情報密度の高い印字をすることができる.ここでQRコードの構造について述べる.図\ref{fig:KDDI_qr_structure}はQRコードの構造を簡易的に表したものである.コードを構成する白黒の正方形はセルと呼ばれる.このセルの組合わせにより情報が表され,QRコード内部には,切り出しシンボル(図\ref{fig:KDDI_qr_structure}\textcircled{2}),タイミングパターン(図\ref{fig:KDDI_qr_structure}\textcircled{3}),アライメントパターン(図\ref{fig:KDDI_qr_structure}\textcircled{4}),フォーマット情報(図\ref{fig:KDDI_qr_structure}\textcircled{5})が存在している.\footnote{QR コードってどういう仕組み?種類や歴史、使用時の注意点などを解説|KDDI トビラ(https://time-space.kddi.com/ict-keywords/20190425/2624)より引用}

\begin{figure}[H]
	\centering
	\includegraphics[width=1.0\linewidth]{pics/KDDI_qr_structure.png}
	\caption[QRコードの構造]{QRコードの構造\protect\footnotemark}
	\label{fig:KDDI_qr_structure}
\end{figure}

\footnotetext{QR コードってどういう仕組み?種類や歴史、使用時の注意点などを解説|KDDI トビラ(https://time-space.kddi.com/ict-keywords/20190425/2624)より引用}


各構造の要素について,それぞれ詳述する.
\paragraph{切り出しシンボル:}位置検出パターン,もしくはファインダーパターンとも呼ばれる.
QRコードの3つの角に配置されており,「黒・白・黒・白・黒」の1:1:3:1:1の比率を持つ正方形パターンをもつ.QRコードの存在検出に使われ,画像中からQRコード領域を切り出す基準となる.

\paragraph{タイミングパターン:}切り出しシンボル同士を結ぶ黒と白が交互に並んだ直線状のパターンであり,最小セルの間隔推定に関わる.


\paragraph{アライメントパターン:}切り出しシンボルより小型の中央に黒点を持つ正方形パターンであり,QRコードのバージョンが上がるほどにQRコード中に含まれる数が増加する。局所的な歪みの補正に使用される.

\paragraph{フォーマット情報:}QRコードの誤り訂正に関する情報を含む.誤り訂正方式の判定にかかわる役割を持ち,正しいデータの復元を可能にする.\\


次に,QRコードの仕様により定められているバージョンについて述べる.QRコードは縦横を構成するセル数毎にバージョンが決められている(図\ref{fig:qr_version}).バージョンは1から40まで設定されており, 例えば, バージョン1では$21 \times 21$セル,  バージョン40では$177 \times 177$セルとなっている.最小サイズであるバージョン1では, 数字で17文字, 英数字で10文字, 漢字で4文字,バイナリで7ビットのデータ表現が可能である.これに対し,最大サイズであるバージョン40では,数字で7,089文字,英数字で4,296文字, 漢字で1,817文字, バイナリで2,953ビットと大容量のデータ表現が可能である.
また, QRコードには3つの角に配置されている特徴的なパターンがある.このパターンをファインダパターンもしくは位置検出パターンと呼ぶ.デコード時にはこのパターンを利用することで, QRコードの位置特定を行う。具体的には, 3つの位置検出パターンを基準として, QRコードの向きや回転状態を検出し, 正しい方向に補正する. その後, 位置検出パターン間の距離を測定することでコードのサイズを認識し,3つのパターンの相対的な配置を基に,QRコード全体の四角形領域を特定する.

図\ref{fig:finder_pattern}に示すように,A,B,Cのいずれの位置においても,白黒部分の幅の比率は1:1:3:1:1となっている.この比率はQRコードが回転した場合でも保持されるため,位置検出パターンの検出結果,及びそれらの位置関係から回転角度を認識することができる.その結果,360°いずれの方向からでも読み取りが可能となり,処理の効率化が実現されている.

\begin{figure}[H]
	\centering
	\includegraphics[width=1.0\linewidth]{pics/qr_version.png}
	\caption[QRコードのバージョン]{QRコードのバージョン\protect\footnotemark}
	\label{fig:qr_version}
\end{figure}

\footnotetext{\protect\cite{QR_version}より引用}

\begin{figure}[H]
	\centering
	\includegraphics[width=1.0\linewidth]{pics/finder_pattern.png}
	\caption[ファインダパターンの原理]{ファインダパターンの原理\protect\footnotemark}
	\label{fig:finder_pattern}
\end{figure}

\footnotetext{\protect\cite{QR_finder_pattern}より引用}


\section{本研究の目的}
本研究の目的は大きく二つに分けられる.一つ目は,QRコードのカラー化によって,面積当たりの情報量を増加させる事である..二つ目は.カラーQRコードの技術を用いて広告媒体等で他のデザインを損なわないような二次元コードを作成することである.










