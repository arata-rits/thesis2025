%/**********************************************************************/
%		第1章:序論
%/**********************************************************************/
\pagenumbering{arabic}
\chapter{序論}
\label{lbl_chptr1}

%/**********************************************************************/
%		研究背景
%/**********************************************************************/

\section{研究背景}
\label{lbl_cp1_haikei}

近年,駅構内の広告や,バスの運賃表,ビルの広告等に多くのデジタルサイネージが使用されており,我々に身近なものとなっている.デジタルサイネージとはディスプレイやタブレット等の電子表示媒体を活用した情報発信システムの総称であり\cite{digital_signage},LEDを利用したものが多い.背景には,1993年に中村修二氏らの研究チームにより青色LEDが発明された事が挙げられる\cite{blue_led}.青色光源の発明によって,赤,緑,青と光の三原色が揃うこととなり,白色を表現することが可能となった.

当初,LEDをサイネージに利用する際には,青色の光を生み出せないために白色を表すことができなかった.そのため文字の表示にオレンジ色のLEDを使用していた(図\ref{signage_orange}).これが現在では,フルカラーの表示が可能となった(図\ref{signage_white}).



\par
\par




\begin{figure}[h]
	\centering
	\includegraphics[width=10cm]{pics/signage_orange}
	\caption[オレンジLEDが使用されたサイネージ]{オレンジLEDが使用されたデジタルサイネージ\protect\footnotemark}
	\label{signage_orange}
\end{figure}
\footnotetext{http://tikutetsuzuki.blog64.fc2.com/blog-entry-664.htmlより引用}

\newpage


\begin{figure}[h]
	\centering
	\includegraphics[width=10cm]{pics/signage_white1}
	\caption[フルカラーLEDが使用されたデジタルサイネージ]{フルカラーLEDが使用されたサイネージ\protect\footnotemark}
	\label{signage_white}
\end{figure}
\footnotetext{https://www.neyagawa-np.jp/uploads/livedoor-blog/neyagawa\_np/imgs/8/4/848dc93a.jpgより引用}

 デジタルサイネージの用途は幅広く,公共交通機関の掲示板から,ショッピングモール内の広告,ビルの広告などに使われている.最近ではペーパーレス化の流れを受け,紙媒体であった商業施設内の顧客用意見箱を電子媒体にしている事例もある(図\ref{ikenbako}).
 
 \begin{figure}[h]
 	\centering
 	\includegraphics[width=5cm]{pics/ikenbako}
 	\caption[電子化された意見箱]{電子化された意見箱\protect\footnotemark}
 	\label{ikenbako}
 \end{figure}
 
 
 
 デジタルサイネージに対する需要の高さはデータからも伺える.図\ref{DSJ_graph}はデジタルサイネージジャパンの来場客数の推移を表したものである.ここでデジタルサイネージジャパン(DSJ)とは,デジタルサイネージの最新動向を伝える目的として毎年開催されている展示会のことである\cite{DSJ}.図\ref{DSJ_graph}から分かるように,2020年以降は新型コロナウイルスの影響で一時的に会場の来場者が大幅に減少したものの,オンライン開催の来場者も含めると,毎年約10万人が参加している.また,2021年から会場での来場者数は年々回復し,2020年以前の来場者数まで回復することが見込まれる.
 
 


\begin{figure}[h]
	\centering
	\includegraphics[width=15cm]{pics/DSJ_graph}
	\caption[デジタルサイネージジャパンの来場客数の推移]{デジタルサイネージジャパンの来場客数の推移\protect\footnotemark}
	\label{DSJ_graph}
\end{figure}


\section{本研究の目的}

本研究では,1.1節で述べたデジタルサイネージの普及を背景として,デジタルサイネージに対する新たな情報の付加手法を提案する.デジタルサイネージは限られた表示領域の中でより多くの情報を表示できることが望ましい.そこで本研究が提案する情報の付加手法が次の二つである.一つ目は,肉眼では視認できないが,カメラで読み取ったときにのみ情報を視認できる情報表示照明装置であり,これをステゴパネルと呼称する.これまで,ステゴパネルの小型化と高解像度化を進めており,更なる機能向上のため異なる点灯方式を用いた仕様の実装を目指す.

二つ目は,同時に複数の情報読み取りが可能なデジタルサイネージであり,これをトリケラパネルと呼称する.この情報付加手法は,光の三原色を個別に読み取ることで複数の情報を提供するものであり,特に,複数のQRコードを一度に表示することを想定して開発する.サイネージ内に複数のQRコードを並べることは,QRコードの誤読み取りや,表示領域の低下につながる.そこで,各QRコードをカラーに変換し,3枚の画像を重ね合わせて表示する.その後,読み取り時に抽出を行うことで各QRコードを別々に読み取る事を目指す.










